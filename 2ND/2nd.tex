\documentclass[a4paper,lithuanian]{article}
% Hello
\usepackage[utf8]{inputenc}
\usepackage[L7x]{fontenc}
\usepackage[lithuanian]{babel}
\usepackage{graphicx}
\usepackage{multirow}
\usepackage{amsmath}
\usepackage{amssymb}
\usepackage{multicol}
\usepackage{listings}
\usepackage{color}
\usepackage{cancel}
\usepackage[normalem]{ulem}

\definecolor{dkgreen}{rgb}{0,0.6,0}
\definecolor{gray}{rgb}{0.5,0.5,0.5}
\definecolor{mauve}{rgb}{0.58,0,0.82}


\lstset{language=Pascal,
  aboveskip=3mm,
  belowskip=3mm,
  showstringspaces=false,
  columns=flexible,
  basicstyle={\small\ttfamily},
  numbers=left,
  numberstyle=\color{black},
  keywordstyle=\color{blue},
  commentstyle=\color{dkgreen},
  stringstyle=\color{mauve},
  breaklines=true,
  breakatwhitespace=true,
  tabsize=4
}

\title{Algoritmų Analizė 2 N.D\\25 Variantas}

\author{
  Ričardas Čubukinas 1910620\\
  Informatika III Kursas 2 Grupė\\
  VU MIF
}

\begin{document}

\maketitle

\section{Uždavinys}
\subsection*{(a)}
\[f(n)=10n\log_2{n}+100n,\quad{}g(n)=n\cdot{}4^{\log_2{}\log_{2}n}.\]
Įsivedę naują kintamąjį $k=\log_2{n}$ paskaičiuokime šiu funkcijų santykio ribą:
\begin{equation}
  \begin{aligned}
    \lim_{n\rightarrow\infty{}}\frac{g(n)}{f(n)} &= \lim_{n\rightarrow\infty{}}\frac{n\cdot{}4^{\log_2{}\log_{2}n}}{10n\log_2{n}+100n}\\
                                                 &= \frac{1}{10}\cdot{}\lim_{n\rightarrow\infty{}}\frac{\cancel{n}\cdot{}\log_{2}^{2}{n}}{\cancel{n}\cdot{}(\log_2{n}+10)}\\
                                                 &=\frac{1}{10}\cdot{}\lim_{k\rightarrow\infty{}}\frac{k^2}{k+10} | : k\\
                                               &=\frac{1}{10}\cdot{}\frac{\overbrace{\lim_{k\rightarrow\infty{}}k}^{\hbox{\infty{}}}}{\underbrace{\lim_{k\rightarrow\infty{}}(1+\frac{10}{k})}_{\hbox{1}}} =\infty{}
  \end{aligned}
\end{equation}

\begin{center}
\begin{tabular}{c c|c c c c c}
  \hline
  \multicolumn{2}{c|}{} & $O$ & $o$ & $\Omega{}$ & $\omega{}$ & \Theta{}\\
  \hline
  \scriptsize{$10n\log_2{n}+100n$}&\scriptsize{$n\cdot{}4^{\log_2{}\log_{2}n}$} & $+$ & $+$ & $-$ & $-$ & $-$\\
  \scriptsize{$n\cdot{}4^{\log_2{}\log_{2}n}$}&\scriptsize{$10n\log_2{n}+100n$} & $-$ & $-$ & $+$ & $+$ & $-$\\
  \hline
\end{tabular}
\end{center}
\subsection*{(b)}
Bandymų ir klaidų metodu randame $N_f$ bei $N_g$,- gauname:
\begin{equation}
\begin{aligned}
&N_f=4516\Rightarrow{}\begin{cases}
f(4516) < 10^6 \\
f(4517) > 10^6
\end{cases}\\
&N_g=6281\Rightarrow{}\begin{cases}
g(6281) < 10^6\\
g(6282) > 10^6
\end{cases}
\end{aligned}
\end{equation}


\section{Uždavinys}
\subsection*{(a)}
Turime funkcijas:
\[f_1(n)=10n^{\log_2{3}},f_2(n)=10n\sqrt{n},f_3(n)=\frac{n^2}{\log_2{n}},f_4(n)=n\cdot{}(\sqrt{2})^{\log_2{n}},f_5(n)=2^{\sqrt{n}}\]

Pirmiausia pertvarkome $f_4$
\[f_4(n)=n\cdot{}(\sqrt{2})^{\log_2{n}}=n\sqrt{n}\]
Funkcijos didėjimo tvarka išsidėstys taip:
\[f_4(n)=n\sqrt{n},f_2(n)=10n\sqrt{n},f_1(n)=10n^{\log_2{3}},f_3(n)=\frac{n^2}{\log_2{n}},f_5(n)=2^{\sqrt{n}}\]
Įsitikinkime:\\
\begin{equation}
  n\sqrt{n}=\Theta{(10n\sqrt{n})}, \text{nes }n\sqrt{n}=0.1 \cdot{}10n\sqrt{n}
\end{equation}
\begin{equation}
  \lim_{n\rightarrow\infty{}}\frac{\cancel{10}n\sqrt{n}}{\cancel{10}n^{\log_2{3}}}
 =\lim_{n\rightarrow\infty{}}\frac{n^{\frac{3}{2}}}{n^{\log_2{3}}}
 =\lim_{n\rightarrow\infty{}}\frac{1}{n^{\frac{2\log_2{3}-3}{2}}}
 =0
\end{equation}
\begin{equation}
  \begin{aligned}
    &\lim_{n\rightarrow\infty{}}\frac{10n^{\log_2{3}}}{\frac{n^2}{\log_2{n}}} =10\cdot{}\lim_{n\rightarrow\infty{}}\frac{n^{\log_2{3}}\log_2{n}}{n^2}\\
    &=10\cdot{}\lim_{n\rightarrow\infty{}}\frac{\log_2{n}}{n^{2-log_2{3}}}
    =10\cdot{}\lim_{n\rightarrow\infty{}}\frac{\frac{1}{n \ln{2}}}{n^{2-log_2{3}}\cdot{}\ln{n}}\\
    &=\frac{10}{\ln{2}}\cdot{}\lim_{n\rightarrow\infty{}}\frac{1}{n^{3-log_2{3}}\cdot{}\ln{n}}
    =\frac{10}{\ln{2}}\cdot{}0=0
  \end{aligned}
\end{equation}
Tebūnie $k = \sqrt{n}$
\begin{equation}
  \begin{aligned}
  \lim_{n\rightarrow\infty{}}\frac{\frac{n^2}{\log_2{n}}}{2^\sqrt{n}}
  &=\lim_{n\rightarrow\infty{}}\frac{n^2}{2^{\sqrt{n}}\cdot{}\log_2{n}}
  =\lim_{k\rightarrow\infty{}}\frac{k^4}{2^{k}\cdot{}\log_2{k^2}}\\
  &=\lim_{k\rightarrow\infty{}}\frac{4k^3}{2^{k}\cdot{}\frac{2}{\ln{(2)}k}\cdot{}\ln{2}}
  =2\cdot{}\lim_{k\rightarrow\infty{}}\frac{k^3}{2^{k}}\\
  &=\frac{6}{\ln{2}}\cdot{}\lim_{k\rightarrow\infty{}}\frac{k^2}{2^{k}}
  =\frac{12}{\ln^{2}{2}}\cdot{}\lim_{k\rightarrow\infty{}}\frac{k}{2^{k}}\\
  &=\frac{12}{\ln^{3}{2}}\cdot{}\lim_{k\rightarrow\infty{}}\frac{1}{2^{k}}
  =\frac{12}{\ln^{3}{2}}\cdot{}0=0
  \end{aligned}
\end{equation}
\subsection*{(b)}
Įstatę $n=16$, gauname:
\[f_1(16)=810,~f_2(16)=640,~f_3(16)=64,~f_4(16)=64,~f_5(16)=16\]
Taigi reikšmės išsidėsto taip:
\[f_5(16) < f_4(16) = f_3(16) < f_2(16) < f_1(16)\]

\subsection*{(c)}
Įstatę $n=2^{16}$, gauname:
\[f_1(2^{16})=10*3^{16},~f_2(2^{16})=10*2^{24},~f_3(2^{16})=2^{28},~f_4(2^{16})=2^{24},~f_5(2^{16})=2^{256}\]
Taigi reikšmės išsidėsto taip:
\[f_4(2^{16}) < f_2(2^{16}) < f_3(2^{16}) < f_1(2^{16}) < f_5(2^{16})\]
\textbf{Ats:.} Funkcijos išsidėstė tokia tvarka (a) 42135 (b) 54321 (c) 42315


\end{document}

\documentclass[a4paper,lithuanian]{article}
% Hello
\usepackage[utf8]{inputenc}
\usepackage[L7x]{fontenc}
\usepackage[lithuanian]{babel}
\usepackage{graphicx}
\usepackage{amsmath}
\usepackage{amssymb}
\usepackage{multicol}
\usepackage{listings}
\usepackage{color}

\definecolor{dkgreen}{rgb}{0,0.6,0}
\definecolor{gray}{rgb}{0.5,0.5,0.5}
\definecolor{mauve}{rgb}{0.58,0,0.82}


\lstset{language=Pascal,
  aboveskip=3mm,
  belowskip=3mm,
  showstringspaces=false,
  columns=flexible,
  basicstyle={\small\ttfamily},
  numbers=left,
  numberstyle=\color{black},
  keywordstyle=\color{blue},
  commentstyle=\color{dkgreen},
  stringstyle=\color{mauve},
  breaklines=true,
  breakatwhitespace=true,
  tabsize=4
}

\title{Algoritmų Analizė 1 N.D}

\author{
  Ričardas Čubukinas 1910620\\
  Informatika III kursas\\
  VU MIF
}

\begin{document}

\maketitle

\section{Uždavinys}
\subsection*{(a)}
$f(n)=\overbrace{\sum_{k=2}^{n-1}k^2}^{\hbox{1}}-{\overbrace{(\sum_{k=3}^{n}k}^{\hbox{2}})^2}$
\begin{enumerate}
  \item{$\sum_{k=2}^{n-1}k^2 = \sum_{k=1}^{n-1}k^2 - 1 = \frac{(k-1)((k-1)+1)(2(k-1)+1)}{6}-1=\frac{k(k-1)(2k-1)}{6}-1$}
  \item{$\sum_{k=3}^{n}k =  \sum_{k=1}^{n}k - 3 = k*\frac{k+1}{2} - 3 = \frac{k^2+k-6}{2}$}
\end{enumerate}
\begin{equation}
  \begin{aligned}
    f(n) & = \sum_{k=2}^{n-1}k^2-(\sum_{k=3}^{n}k)^2\\
         & = \frac{k(k-1)(2k-1)}{6}-1-{(\frac{k^2+k-6}{2})}^2\\
         & = \frac{-3k^4-2k^3+27k^2+38k}{12}
  \end{aligned}
\end{equation}
\subsection*{(b)}
Kadangi funkcija neauga eksponentiškai, rasime asimptotiką, t.y $a$ ir $b$ tokias kad, $f(n)\sim{}an^b$, kai $n\rightarrow{}\infty{}$
\begin{equation}
\lim_{n\rightarrow{}\infty{}}\frac{\overbrace{-3n^4}^{}-2n^3+27n^2+38n}{12an^b} = 1
\end{equation}
\begin{equation}
  \frac{-3*n^4}{12an^b} \Rrightarrow{} \begin{cases} a = -\frac{1}{4} \\ b = 4  \end{cases}
\end{equation}
\begin{equation}
f(n)\sim{}-\frac{n^4}{4}
\end{equation}
\section{Uždavinys}
\subsection*{(a)}
Duotas sveikų skaičių masyvas $A[1 : n]; c = 2$.
\begin{multicols}{2}
  \begin{lstlisting}
for j := 1 to n do
  C[j] := 0
  i := n
    while i >= j do
      C[j] := C[j] + A[i]
      i := i - 1
  if C[j] < 0 then C[j] := 0
  \end{lstlisting}
  \columnbreak
  $2(n+1)$\\
  $n$\\
  $n$\\
  $\sum_{i=1}^{n}i + n$\\
  $\sum_{i=1}^{n}i$\\
  $\sum_{i=1}^{n}i$\\
  $2n$\\
\end{multicols}
\begin{equation}
  \begin{aligned}
    L(n)&=2(n+1)+n+n+3\sum_{i=1}^{n}i+n+2n\\
        &=2n + 2 + n + n  + \frac{3n(n+1)}{2} + n + 2n\\
        &=\frac{3n^2+17n+4}{2}
  \end{aligned}
\end{equation}
\subsection*{(b)}
\begin{equation}
L(n)\sim{}a*n^b
\end{equation}
\begin{equation}
  \lim_{n\rightarrow\infty{}}\frac{f(n)}{g(n)}=1
\end{equation}
\begin{equation}
  \lim_{n\rightarrow\infty{}}\frac{\overbrace{3n^2}^{}+17n+4}{2an^b}=1
\end{equation}
\begin{equation}
  \frac{3n^2}{2an^b} \Rrightarrow{} \begin{cases} a = \frac{3}{2} \\ b = 2  \end{cases}
\end{equation}
\begin{equation}
L(n)\sim{}\frac{3n^2}{2}
\end{equation}
\subsection*{(c)}
Pasirinksime „blogiausius“ duomenis kai $n = c = 2$. Pakanka jog paskutinis masyvo skaičius būtų neigiamas ir masyvo skaičių suma būtų neigiama, tarkime $A = [-2, -5]$.
Kadangi $L(2) = \frac{3\cdot{4}+17\cdot{2}+4}{2}=25$, tai bus atlikti 25 žingsniai:
\begin{lstlisting}
j := 1
1 <= 2? TRUE
C[1] = 0
i := 2
2 >= 1? TRUE
C[1] := -5
i := 1
1 >= 1? TRUE
C[1] := -7
i := 0
0 >= 1? FALSE
-7 < 0? TRUE
C[1] := 0
j := 2
2 <= 2? TRUE
C[2] := 0
i := 2
2 >= 2? TRUE
C[2] = -5
i := 1
1 >= 2? FALSE
-5 < 0? TRUE
C[2] = 0
j := 3
3 <= 2? FALSE
\end{lstlisting}
\subsection*{(d)}
Skaičiuojant programos vykdymo laiką kiekvieną komandos eilutę „dauginsime“ iš konstantos $c_i$ atitinkančios tos eilutės vykdymo laiką:
\begin{multicols}{2}
  \begin{lstlisting}
for j := 1 to n do
  C[j] := 0
  i := n
    while i >= j do
      C[j] := C[j] + A[i]
      i := i - 1
  if C[j] < 0 then C[j] := 0
  \end{lstlisting}
  \columnbreak
  $c_12(n+1)$\\
  $c_2n$\\
  $c_3n$\\
  $c_4(\sum_{i=1}^{n}i + n)$\\
  $c_5\sum_{i=1}^{n}i$\\
  $c_6\sum_{i=1}^{n}i$\\
  $c_72n$\\
\end{multicols}
Turime $\sum_{i=1}^{n}i=\frac{n(n+1)}{2}$. Apskaičiuojame $T(n)$:
\begin{equation}
  \begin{aligned}
    T(n) &= c_{1}2(n+1)+n(c_2+c_3+c_4+2c_7)+(c_4+c_5+c_6)(\frac{n^2}{2}+\frac{n}{2})\\
         &= \frac{c_4+c_5+c_6}{2}n^2 + (c_2 + c_3 + \frac{3c_4}{2} + \frac{c_5}{2} + \frac{c_6}{2} + 2c_7)n + 2c_1\\
         &= An^2 + Bn + C
  \end{aligned}
\end{equation}
Taigi, $T(n) = \Theta{(n^2)}$. Kadangi $\Theta{}(n^d)$, kai $n\rightarrow{}\infty{}$ programos vykdymo laiko eilė $d = 2$. Kitaip tariant tai yra „kvadratinio sudėtingumo“ algoritmas.
\end{document}

\documentclass{VUMIFInfKursinis}
\usepackage{algorithmicx}
\usepackage[linesnumbered,ruled,vlined]{algorithm2e}
\usepackage{caption}
\usepackage{amsfonts}
\usepackage{amsmath}
\usepackage{bm}
\usepackage{color}
\usepackage{graphicx}
\usepackage{xcolor}
\usepackage{hyperref}
\usepackage{url}


% Titulinio aprašas
\university{Vilniaus universitetas}
\faculty{Matematikos ir informatikos fakultetas}
\institute{Informatikos institutas}
\department{Informatikos katedra}
\papertype{Projektinis Darbas}
\title{Visų maksimalių grafo klikų generavimas}
\titleineng{Generating all maximal cliques}
\status{3 kurso 2 grupės studentas}
\author{Ričardas Čubukinas}
\supervisor{Asist., Dr. Valdas Dičiūnas}

\makeatletter
\def\BState{\State\hskip-\ALG@thistlm}
\makeatother

\date{Vilnius \\ \the\year}

% Nustatymai
\setmainfont{Palemonas}
\bibliography{bibliografija} 
\renewcommand{\algorithmcfname}{Algoritmas}

\begin{document}
\maketitle

\tableofcontents

\sectionnonum{Problemos formulavimas}
\begin{itemize}
  \item{\textbf{Duota:} Neorientuotas grafas $G=(V,E)$, turintis $n$ viršūnių ir $m$ briaunų.}
  \item{\textbf{Rasti:} Visas maksimalias grafo G klikas, t.y. tokius viršūnių aibės poaibius kurie grafe $G$ sudaro pilną pografį ir kurių nebegalima praplėsti taip, kad vėl gautųsi pilnas pografis.}
\end{itemize}

Realizuoti visų grafo klikų paieškos su grįžimu atgal („backtracking“) algoritmą ir ištirti jo
sudėtingumą:
\begin{enumerate}
  \item{Teoriškai.}
  \item{Praktiškai kaip priklauso nuo grafo viršūnių skaičiaus $n$ ir briaunų skaičiaus $m$.}
\end{enumerate}

\section{Algoritmas}
\begin{algorithm}[H]
  \DontPrintSemicolon
  \caption{Bron Kerbosch}
  \textbf{Duomenys:} Neorientuotas grafas $G=(V,E)$ \;
  \textbf{Rezultatas:} Visos maksimalios grafo $G$ klikos \;
  \SetKwFunction{FMain}{generate\_all\_cliques}
  \SetKwProg{Fn}{Function}{:}{}
  \Fn{\FMain{$G$, $R$, $X$}}{
    \If{$G$ and $X$ \textbf{not} $\emptyset$}{
      print($G$)  \;
      \ForEach{vertex $v$ $\in$ $G$}{
       generate\_all\_cliques($G$ $\cap$ $N(v)$, $R$ $\cup$ $\{v\}$, $X$ $\cap$ $N(v)$) \;
       $G \gets G \setminus {v}$ \;
       $X \gets X \setminus {v}$ \;
      }
    }
  }
\end{algorithm}

% Eksperimentavimas su skirtingais duomenimis vykdymo laikas ats ir t.t
\section{Eksperimentai}

% Teoriškai išanalizuota ir praktiškai
\section{Sudėtingumo analizė}

\section{Išvados}

\section{Programos naudojimo instrukcija}

\printbibliography[heading=bibintoc] % Literatūros šaltiniai aprašomi

%\appendix  % Priedai

\end{document}

\documentclass{VUMIFInfKursinis}
\usepackage{algorithmicx}
\usepackage[linesnumbered,ruled,vlined]{algorithm2e}
\usepackage{caption}
\usepackage{amsfonts}
\usepackage{amsmath}
\usepackage{bm}
\usepackage{color}
\usepackage{graphicx}
\usepackage{xcolor}
\usepackage{hyperref}
\usepackage{url}


% Titulinio aprašas
\university{Vilniaus universitetas}
\faculty{Matematikos ir informatikos fakultetas}
\institute{Informatikos institutas}
\department{Informatikos katedra}
\papertype{Projektinis Darbas}
\title{Visų maksimalių grafo klikų generavimas}
\titleineng{Generating all maximal cliques}
\status{3 kurso 2 grupės studentas}
\author{Ričardas Čubukinas}
\supervisor{Asist., Dr. Valdas Dičiūnas}

\makeatletter
\def\BState{\State\hskip-\ALG@thistlm}
\makeatother

\date{Vilnius \\ \the\year}

% Nustatymai
\setmainfont{Palemonas}
\bibliography{bibliografija} 
\renewcommand{\algorithmcfname}{Algoritmas}

\begin{document}
\maketitle

\tableofcontents

\sectionnonum{Problemos formulavimas}
\begin{itemize}
  \item{\textbf{Duota:} Neorientuotas grafas $G=(V,E)$, turintis $n$ viršūnių ir $m$ briaunų.}
  \item{\textbf{Rasti:} Visas maksimalias grafo G klikas, t.y. tokius viršūnių aibės poaibius kurie grafe $G$ sudaro pilną pografį ir kurių nebegalima praplėsti taip, kad vėl gautųsi pilnas pografis.}
\end{itemize}

Realizuoti visų grafo klikų paieškos su grįžimu atgal („backtracking“) algoritmą ir ištirti jo
sudėtingumą:
\begin{enumerate}
  \item{Teoriškai.}
  \item{Praktiškai kaip priklauso nuo grafo viršūnių skaičiaus $n$ ir briaunų skaičiaus $m$.}
\end{enumerate}

\section{Algoritmas}
Bron-Kerbosch algoritmo forma yra rekursinis grįžtamasis(„backtracking“) algoritmas, kuriuo ieškoma visų maksimalių klikų duotame grafe $G$. Esant trims nesujungtoms viršūnių aibėms $R$, $P$ ir $X$, randamos maksimalios klikos, į kurias įeina visos $R$ viršūnės, kai kurios $P$ viršūnės ir nė viena $X$ viršūnė. Kiekvieno algoritmo iškvietimo metu $P$ ir $X$ yra nesujungtos aibės, kurių sąjungą sudaro tos viršūnės, kurios, pridėjus jas prie R, sudaro kliką. Kitaip tariant, $P \cup X$ yra aibė viršūnių, kurios yra sujungtos su kiekvienu $R$ elementu.

Rekursija pradedama nustatant, kad $R$ ir $X$ yra tuščia aibė, o $P$ - grafo viršūnių aibė. Per kiekvieną rekursinį iškvietimą algoritmas paeiliui nagrinėja $P$ viršūnes; jei tokių viršūnių nėra, jis arba praneša, kad $R$ yra maksimali klika (jei $X$ tuščia), arba grįžta atgal. Kiekvienai viršūnei $v$, pasirinktai iš $P$, algoritmas atlieka pasikartojantį iškvietimą, per kurį $v$ pridedama prie $R$ ir per kurį $P$ ir $X$ apsiriboja $v$ kaimynų aibe $N(v)$, kurioje surandami ir pranešami visi $R$ klikos plėtiniai, kuriuose yra $v$. Tada $v$ perkeliama iš $P$ į $X$, kad į ją nebūtų atsižvelgta būsimuose klikose, ir tęsiamas darbas su kita $P$ viršūne. 

Tačiau aprašyta pagrindinė algoritmo forma yra neefektyvi, kai grafai turi daug nemaksimalių klikų: kiekvienai klikai, nesvarbu, ar ji maksimali, ar ne, atliekamas rekursinis iškvietimas. Siekdami sutaupyti laiko ir leisti algoritmui greičiau grįžti į paieškos šakas, kuriose nėra maksimalių klikų, Bron ir Kerbosch įvedė algoritmo variantą, apimantį "atsparos tašką" $u$, parinktą iš iš $P \cup X$. Bet kurioje maksimalioje klikoje turi būti arba $u$, arba vienas iš jos neesančių kaimynų, nes priešingu atveju klika galėtų būti padidinta į ją įtraukus $u$. Todėl kiekviename algoritmo rekursyviame iškvietime į $R$ pridedamos viršūnės $v$ pasirinkimo variantai turi būti tikrinami tik $u$ ir jos ne kaimynės.

Translated with www.DeepL.com/Translator (free version)

\begin{algorithm}[H]
  \DontPrintSemicolon
  \caption{Bron Kerbosch su atsparos tašku}
  \textbf{Duomenys:} Neorientuotas grafas $G=(V,E)$ \;
  \textbf{Rezultatas:} Visos maksimalios grafo $G$ klikos \;
  $P \gets V$ \;
  $N \gets \emptyset$ \;
  \ForEach{briauna $e_i, e_j$ $\in$ E}{
    $N[e_i] \gets N[e_i] + e_j$ \;
    $N[e_j] \gets N[e_j] + e_i$ \;
  }
  \SetKwFunction{FMain}{generuoti\_maksimalias\_klikas}
  \SetKwProg{Fn}{Function}{:}{}
  \Fn{\FMain{$P$, $R$, $X$}}{
    \If{$P$ and $X$ \textbf{not} $\emptyset$}{
      paskelbti R kaip maksimalią kliką  \;
    }
      parenkame atsparos tašką - briauną $u$ iš $P \cap X$ \;
      \ForEach{briauna $v$ $\in$ $P \setminus N(u)$}{
       \FMain{$P$ $\cap$ $N(v)$, $R$ $\cup$ $\{v\}$, $X$ $\cap$ $N(v)$} \;
       $P \gets P \setminus \{v\}$ \;
       $X \gets X \setminus \{v\}$ \;
      }
    
  }
\end{algorithm}

% Eksperimentavimas su skirtingais duomenimis vykdymo laikas ats ir t.t
\section{Eksperimentai}

% Teoriškai išanalizuota ir praktiškai
\section{Sudėtingumo analizė}

\section{Išvados}

\section{Programos naudojimo instrukcija}

\printbibliography[heading=bibintoc] % Literatūros šaltiniai aprašomi

%\appendix  % Priedai

\end{document}
